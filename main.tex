\documentclass{article}
\usepackage[utf8]{inputenc}
\usepackage{graphicx}
\usepackage{csquotes}
\setlength{\parindent}{0pt}
\setlength{\parskip}{1em}
\renewcommand{\baselinestretch}{1.8}


\usepackage[
    backend=biber,
    style=authoryear,
  ]{biblatex}
 
\addbibresource{sample.bib}

\title{Why Just Speaking Back Won't Work - A Reply to Katharine Gelber's 'Speaking Back'}
\author{Jan Jugueta - PHIL702 Paper Plan}
\date{3 October 2019}

\begin{document}

\maketitle

\section{The Purpose of My Paper}

In ‘’Speaking Back’: The Likely Fate of Hate Speech Policy in the United States and Australia’, Katherine Gelber advances the idea that hate speech can be combatted with a policy of speaking back. Hate speech is detrimental to the targeted individual not only because it marginalises and disempowers them, but it also hinders their speech ability. The ability to speak, Gelber claims, is central to an individual living a meaningful life. Gelber argues that such a policy would provide targets of hate speech with ‘institutional, educational, and material support to enable them to speak back’ (\cite{gelber}, p. 51)  ameliorating the effects of hate speech. According to Gelber, a speaking back policy is not speech-restricting but rather speech enhancing (\cite{gelber}, p. 55), thereby not infringing on an individual’s First Amendment rights. Gelber conceptualises the speaking back policy as an ‘explicitly individually based policy idea’ (ibid). Focussing specifically on racially charged hate-speech acts, I will argue that Gelber’s proposal of a speaking back policy fails to sufficiently address the larger structural power imbalances present in society that fuel hate-speech acts. I will also argue that Gelber’s positioning of the speaking back policy centred on the individual is inconsistent with the potential solutions she proposes later in her chapter.

\section{The Structure of My Argument}

To explain why I believe that Gelber’s policy of speaking back is deficient in the fight against hate speech, my paper will address the following points:

\begin{itemize}
    \item Although conceived of as speech-enhancing and not speech-restricting, a speaking back policy made into law would be seen as promoting a particular viewpoint, thereby raising First Amendment concerns. Gelber herself admits this is a possibility. Instead, Gelber shifts the argument by stating the speaking back policy need not require that targets speak back but would support them if they wish to do so (ibid, p. 59). I would argue that if a target were to take up the support, it would then be seen as promotion of a viewpoint.
    \item Gelber states early in her chapter that the speaking back policy is an ‘explicitly individually based policy idea’ (ibid), but her proposed solutions of using government or other institutions to do the speaking back is contradictory since the individual is not the agent speaking back.
    \item Gelber proposes members of a governmental subsidised anti-discrimination institution could speak back to hate speakers if they themselves are not targets of hate-speech. A paradox could then arise where they do become the targets of hate speech, thereby creating a situation where the government are guilty of viewpoint discrimination and violate the First Amendment.
    \item Gelber does not sufficiently address how speaking back will ameliorate the effects of hate speech. It is not enough to rely on the assumption that speaking back to contradict the hate speaker’s viewpoint will nullify the perlocutionary effects brought about by hate speech.
    \item The idea of a speaking back policy assumes that the target of a hate-speech act can speak back to the hate speaker with an equivalent force. Gelber’s proposal fails to recognise that in the context of racial hate-speech acts, the historical relationship between whites and non-whites in countries like Australia and the United States has been one of hierarchy and domination.
    \item Whilst a speaking back policy may be more achievable in an Australian context due to a lack of a bill of rights, it only reinforces the idea that such a policy would not work in the United States because of the First Amendment.
\end{itemize}

Although I agree with the general sentiment that hate speech should be combatted, the First Amendment in the United States is an impediment to tackling hate speech. I understand Gelber’s proposal to be a pragmatic approach, but its descriptive character is why it is deficient. In my opinion, not enough emphasis is placed on why hate speech is harmful to a target. I would instead focus towards the perlocutionary function of hate speech and gain an understanding of why it is harmful. After such an understanding is gleaned, perhaps an alternative solution could be presented.

\section{Relevant Papers}

To assist my analysis of Gelber’s proposal in ‘Speaking Back’, I will use three other papers that strengthen my argument as to why I think speaking back alone is insufficient in combatting hate speech.

\subsection*{\textit{Slurs} by Adam M. Croom}

Adam Croom’s article examines the linguistic properties of slurs. Croom defines a slur as ‘a disparaging remark or a slight that is usually used to deprecate certain targeted members' (\cite{croom}, p. 343). The relevance that this article has on my paper is that it provides an insight into why hate-speech acts can be harmful to the target. The paper will utilise Croom’s nuanced investigation into the semantic and pragmatic function of slurs, along with his understanding of slurs’ basis in historical acts and power relations (ibid, p. 354). 

\subsection*{\textit{How to do things with words} by John Langshaw Austin}

Since a major criticism I have with Gelber’s proposal centres on the lack of discussion surrounding the perlocutionary effects that hate-speech acts have on their targets, Austin’s \textit{How to do things with words} can help provide an understanding on the function of perlocutionary acts. Austin describes perlocutionary acts as ‘the achieving of certain effects by saying something’ (\cite{austin}, p. 121). Focussing on the perlocutionary effects of hate-speech acts, that is, the harm caused to the target, will bring into question whether the act of speaking back can ameliorate the effects caused by hate speech.

\subsection*{\textit{The Harm in Hate Speech} by Jeremy Waldron}

In \textit{The Harm in Hate Speech}, Jeremy Waldron makes a case in favour of hate speech laws. Waldron argues that hate speech laws should be implemented for the protection of individual human dignity. Like Gelber, Waldron believes that the individual is entitled to be a ‘member of society in good standing, whose membership does not disqualify him or her from ordinary social interaction' (\cite{waldron}, p. 105). Waldron, unlike Gelber, targets change at the legislative level. Although both Waldron and Gelber’s proposals can raise First Amendment concerns, Waldron’s advocacy for the protection of human dignity, if successfully implemented, could be more effective in combating hate speech. 

\newpage

\printbibliography

\end{document}
